% !TeX root = hw.tex
\section{环论}
\setcounter{pb}{8}
\begin{problem}
    \label{pb: ring, 8}
    证明 $ \mathbb{Z}[x] $ 的任一个主理想非极大。
\end{problem}

\begin{lemma}
    If $f(x)$ is irreducible in $\mathbb{Z}[x]$, then only the following two possibilities exist: 
    \begin{enumerate}
        \item $f(x)$ is a prime in $\mathbb{Z}$;
        \item $f(x)$ is irreducible in $\mathbb{Q}[x]$.
    \end{enumerate}
\end{lemma}
\begin{proof}[Proof.]
    In the case $\deg f(x)=0$: it turns out that it is a prime. In the case $\deg f(x)\geq1$, we have $f(x)=c f_0(x)$, 
    for a primitive polynomial $f_0(x)$ and $c\in\mathbb{Z}$. We assume $c\in\{\pm1\}$ by irreduciblity. 
    For a primitive polynomial, it is irreducible in $\mathbb{Z}[x]$ if and only if it is irreducible in $\mathbb{Q}[x]$.
\end{proof}

\begin{solution}
    我们假设 $\mathbb{Z}[x]$ 中的主理想 $\langle f(x) \rangle $ 是极大的, 则这是极大主理想, 即 $f(x)$ 是不可约元. 
    而不可约元只有两种: $f(x)=p\in\mathbb{Z}$ 是素数; $f(x)$ 是 $\mathbb{Q}[x]$ 中不可约多项式. 
    当 $f(x)=p$ 是素数, 有 $\mathbb{Z}[x]/\langle p \rangle=\mathbb{F}_{p}[x]$ 是整环不是域, 所以主理想非极大. 
    当 $f(x)$ 是 $\mathbb{Q}[x]$ 中不可约多项式, 有 $\langle f(x) \rangle \subseteq \langle f(x),p \rangle \subsetneq \mathbb{Z}[x]$ 其中 $p$ 是素数;所以主理想非极大.
\end{solution}

\subsection{更多习题}
\setcounter{pb}{13}

\begin{problem}
    Show that the ideal $(3, x^3-x^2 + 2x- 1)$ in $\mathbb{Z}[x]$ is not principal.
\end{problem}

\begin{solution}
    This ideal is maximal:
        \[
            \mathbb{Z}[x]/(3, x^3-x^2 + 2x- 1)\cong({\mathbb{F}_3[x]})/( x^3-x^2 + 2x- 1),
        \]
    since $x^3-x^2 + 2x- 1$ has no roots in $\mathbb{F}_3$; it is irreducible. 
    From 题目~\ref{pb: ring, 8}, we know it is not principal.
\end{solution}

\section{模论}
\subsection{第五章}
\setcounter{pb}{4}
\begin{problem}
    设 $ \mathbb{Q} $ 为有理数域,$ M $ 和 $ M' $ 是两个左 $ \mathbb{Q} $ 模. 证明:若 $ \eta : M \to M' $ 是一个加法群同构,则 $ \eta $ 也是一个 $ \mathbb{Q} $ 模同构. 
    ($*$ 如果用实数域 $ \mathbb{R} $ 替代 $ \mathbb{Q} $,问这个命题是否成立?)
\end{problem}

\begin{solution}
    对 $ x,y\in M$,$ p_{1}/q_{1},p_{2}/q_{2}\in\mathbb{Q}$ 有 $x/q_{1}, y/q_{2}\in M$ 且
        \[
            \eta(r x+s y)=\eta\Big(\sum_{i=1}^{p_{1}}\frac{x}{q_{1}}+\sum_{j=1}^{p_{2}}\frac{y}{q_{2}}\Big)
            =\sum_{i=1}^{p_{1}}\eta\Big(\frac{x}{q_{1}}\Big)+\sum_{j=1}^{p_{2}}\eta\Big(\frac{y}{q_{2}}\Big);
        \]
    另一方面, 
        \[
            \eta(x)=\eta\Big(\sum_{i=1}^{q_{1}}\frac{x}{q_{1}}\Big)=\sum_{i=1}^{q_{1}}\eta\Big(\frac{x}{q_{1}}\Big)=q\eta\Big(\frac{x}{q_{1}}\Big)\implies \eta\Big(\frac{x}{q_{1}}\Big)=\frac{1}{q_{1}}\eta\Big(\frac{x}{q_{1}}\Big),
        \]
    综上所述, 
        \[
            \eta(r x+s y)=\frac{p_{1}}{q_{1}}\eta(x)+\frac{p_{2}}{q_{2}}\eta(y).
        \]
    此时, $\eta$ 成 $\mathbb{Q}$ 模同构.
    \par 对 $\mathbb{R}$ 模, 命题不成立: 视 $\mathbb{R}$ 为 $\mathbb{Q}$ 模, 依 Zorn 引理可取一组基 $\{e_{i}\}_{i\in I}$. 
    定义 $\eta\colon \mathbb{R}\to\mathbb{R}, \sum_{\text{有限}}r_{i}e_{i}\mapsto\sum_{\text{有限}}r_{i}\lambda_{i}e_{i}$. 
    则 $\eta$ 的矩阵形式为对角矩阵 $\operatorname{diag}(\lambda_{i})_{i}$; 命这个对角矩阵可逆且 $\lambda_{i}$ 不全相同, 则 $\eta$ 是加法群同构, 但不是 $\mathbb{R}$ 模同构.
\end{solution}

\setcounter{pb}{19}
\begin{problem}
    将 $ \mathbb{Z}/(n) $ 看作 $ \mathbb{Z} $ 模,问下列模是否可写成两个非零子模的直和:
    \begin{enumerate}[label=(\roman*)]
        \item $ \mathbb{Z}/(p^e), p $ 为素数,$ e \geq 1 $;
        \item $ \mathbb{Z}/(n), n = p_1^{e_1} \cdots p_r^{e_r}, p_1, \dots, p_r $ 为不同的素数,$ e_i \geq 1, i = 1, \dots, r $.
    \end{enumerate}
\end{problem}

\begin{solution}
    作为 $\mathbb{Z}$ 模的直和分解, 即分解为 Abel 群的直和.
    \par 若有分解, 分析子群的阶数, 可知分解必然形如
        \[
            \mathbb{Z}/(p^{e})=\mathbb{Z}/(p^{r})\oplus\mathbb{Z}/(p^{s}),
        \]
    其中 $r+s=e$. 此时, 左边有 $p^{e}$ 阶元, 而右边元素阶数最大为 $\max\{p^{r},p^{s}\}$; 应当相等, 故 $r=e$ 或者 $s=e$, 于是 $\mathbb{Z}/(p^{e})$ 的分解一定是平凡的.
    \par 根据中国剩余定理, 有分解
        \[
            \mathbb{Z}/(n)=\mathbb{Z}/(p_{1}^{e_{1}})\oplus\mathbb{Z}/(p_{2}^{e_{2}}\cdots p_{r}^{e_{r}}),
        \]
    这里两个子模均非零. 
\end{solution}

\setcounter{pb}{20}
\begin{problem}
    证明:$ \mathbb{Q} $ 作为 $ \mathbb{Z} $ 模,它的任一有限生成的子模是循环模. 由此证明,$ \mathbb{Q} $ 不是一个自由 $ \mathbb{Z} $ 模. 
\end{problem}

\begin{solution}
    设 $M=\langle p_{1}/q_{1},\cdots,p_{n}/q_{n} \rangle \subseteq \mathbb{Q}$ 是有限生成子模, 其中 $p_{i},q_{i}$ 互素. 
    令 $q \coloneqq \operatorname{lcm}(q_{1},\cdots,q_{n})$, 则
        \[
            M=\left\langle \frac{p_{1}r_{1}}{q},\cdots,\frac{p_{n}r_{n}}{q} \right\rangle,\quad r_{i}\coloneqq \frac{q}{q_{i}}.
        \]
    可见 $M=\bigl\langle \frac{\gcd({p_{1}r_{1}},\cdots,{p_{n}r_{n}})}{q} \bigr\rangle$, 是循环模.
    \par 假设 $\mathbb{Q}$ 是自由 $\mathbb{Z}$ 模, 而前述性质表明, 它没有秩 $>1$ 的有限生成子模, 故只能是秩为 $1$ 的自由模, 这不可能: 若是秩为 $1$ 的自由 $\mathbb{Z}$ 模, 则正的部分应有最小元.
\end{solution}

\subsection{第六章}

\subsection{补充}

\setcounter{pb}{3}
\begin{problem}
    设 $ R $ 是交换环,$ I $ 为 $ R $ 的理想. $ M $ 是有限生成的 $ R $ 模,$ \varphi \in \operatorname{End}_R(M) $. 
    \begin{enumerate}[label=(\arabic*)]
        \item 如果 $ M \subseteq IM $,证明:存在 $ f(x) = x^n + a_1 x^{n-1} + \dots + a_n $($ a_1, \dots, a_n \in I $),使得
              \[
                  f(\varphi) (= \varphi^n + \varphi^{n-1} a_1 + \dots + \operatorname{id} |_M a_n) = 0 \in \operatorname{End}_R(M)
              \]
        \item Nakayama 引理 (重要): $ R $ 的所有极大理想的交称为 $ R $ 的 Jacobson 根,记为 $ J(R) $. 如果 $ M J(R) = M $,证明 $ M = (0) $. 
    \end{enumerate}
\end{problem}

\begin{vain}
    \dots
    \par 我们证明 $\left.{\mathrm{id}}\right|_{M}^{M}=0$, 从而 $M=(0)$; 简记 $\mathrm{id}=\left.{\mathrm{id}}\right|_{M}^{M}$. 由 (1), 有 $f(x) = x^n + a_1 x^{n-1} + \dots + a_n,(a_{i}\in I)$ 使得  
        \[
            f(\mathrm{id})=\Bigl(1+\sum_{i=1}^{n}a_{i}\Bigr)\mathrm{id}=0.
        \]
    记 $a=\sum_{i=1}^{n}a_{i}\in I$, 下证明 $1+a$ 是 $R$ 中可逆元: 如果不然, 则主理想 $ \langle 1+a \rangle  $ 是非平凡的, 
    于是有某极大理想 $I_{0}\supseteq \langle 1+a \rangle$, 但是 $a\in J(R)\implies a\in I_{0}$, 所以 $1=(1+a)-a\in I_{0}$, 
    矛盾: 极大理想不是平凡理想, 不应该有 $1$. 综上所述, $1+a$ 是可逆的, 故 $\mathrm{id}=0$.
\end{vain}

\setcounter{pb}{6}
\begin{problem}
    设 $ \varphi: M \to M $ 是 $ R $ 的模同态,且 $ \varphi \varphi = \varphi $. 
    证明:
    \[
        M = \ker \varphi \oplus \operatorname{im} \varphi.
    \]
\end{problem}

\begin{solution}
    对 $x\in M$, 
        \[
            x=\underbracket{x-\varphi x}_{\in\ker \varphi}+\underbracket{\varphi x}_{\in \operatorname{im}\varphi}.
        \]
    另一方面, $x\in\ker\varphi\cap \operatorname{im}\varphi\implies x=0$; 因为 $x=\varphi y\implies 0=\varphi x=\varphi y\implies 0=x$. 
    综上所述, 分解 $M=\ker\varphi+\operatorname{im}\varphi$ 是直和分解, 即 $M=\ker\varphi\oplus\operatorname{im}\varphi$. 
\end{solution}
    
\setcounter{pb}{10}
\begin{problem}
    Determine $\operatorname{End}(\mathbb{Q},+,0)$.
\end{problem}

\begin{solution}
    It is isomorphic to the ring $\mathbb{Q}$:
        \begin{equation}
            \label{eqn: pb.grp.more.9}
            \operatorname{End}(\mathbb{Q},+,0) \to \mathbb{Q},\; f\mapsto f(1). 
        \end{equation}
    It suffices to show the morphism is injective.  Suppose $f(1)=g(1)$, then $\forall m\in\mathbb{Z},\forall n\geq1  $, 
        \[
            \begin{split}
                & f(1)=n\cdot f\Big(\frac{1}{n}\Big)=n\cdot g\Big(\frac{1}{n}\Big)=g(1);\\
                & f\Big(\frac{m}{n}\Big)=m f\Big(\frac{1}{n}\Big)=m g\Big(\frac{1}{n}\Big)=g\Big(\frac{m}{n}\Big).
            \end{split}
        \]
    Thus $f=g$, \eqref{eqn: pb.grp.more.9} is injective. 
    It keeps addition, and also multiplication:
        \[
            f\Big(\frac{m}{n}\Big)=\frac{m}{n}f(1)\implies(f\circ g)(1)=f( g(1) )=f(1) g(1).
        \]
\end{solution}

\section{域论}
\subsection{第七章}

\setcounter{pb}{2}
\begin{problem}
    设 $ K/F $ 为一有限扩张,$ \alpha \in K $ 是 $ F $ 上一个 $ n $ 次元素,证明 $ n \mid [K : F] $.
\end{problem}

\begin{solution}
    有中间域 $F(\alpha) $, 于是 
        \[
            [K:F]=[K:F(\alpha)][F(\alpha):F]=[K:F(\alpha)]\times n
        \]
    因为 $[F(\alpha):F]=\deg\alpha=n$.
\end{solution}

\setcounter{pb}{4}
\begin{problem}
    设 $ K $ 为 $ F $ 上域扩张. 证明:如果 $ u \in K $ 是 $ F $ 上代数元且次数为奇数,则 $ u^2 $ 也是 $ F $ 上奇次数代数元且 $ F(u) = F(u^2) $.
\end{problem}

\begin{solution}
    设 $u$ 在 $F$ 上的极小多项式为 $p_{u}(x)=x^{2n+1}+\sum_{j=0}^{2n}\lambda_{j}x^{j}$. 
    则 
        \[
            p(u)=0,\quad p(x)\coloneqq \Big((u^{2})^{n}+\sum_{j=1}^{n}\lambda_{2j-1}u^{2j-2}\Big)x+\sum_{j=0}^{n}\lambda_{2j}(u^{2})^{j}\in F(u^{2})[x].
        \]
    所以, $u$ 在 $F(u^{2})$ 上的极小多项式的次数不超过 $1$(也就只能是 $1$), 故 $[F(u):F(u^{2})]=1$, 这就是 $F(u)=F(u^{2})$. 由 
        \[
            [F(u):F]=[F(u):F(u^{2})] [F(u^{2}):F]
        \]
    可知 $u^{2}$ 在 $F$ 上次数也是奇数次.
\end{solution}

\setcounter{pb}{6}
\begin{problem}
    求下列扩域的一基:
    \begin{enumerate}[label=(\roman*)]
        \item $ K = \mathbf{Q}(\sqrt{2}, \sqrt{3}) $;
        \item $ K = \mathbf{Q}(\sqrt{3}, \sqrt{-1}, \omega) $, 其中 $ \omega = \frac{1}{2}(-1 + \sqrt{-3}) $.
    \end{enumerate}
\end{problem}

\begin{solution}
    有 $K=\operatorname{span}_{\mathbb{Q}}(1,\sqrt{2},\sqrt{3},\sqrt{6})$, 一组基是 $\{1,\sqrt{2},\sqrt{3},\sqrt{6}\}$.
    \par 有 $K=\operatorname{span}_{\mathbb{Q}}(1,\sqrt{3},\sqrt{-1},\sqrt{-3})$, 一组基是 $\{1,\sqrt{3},\sqrt{-1},\sqrt{-3}\}$.
\end{solution}

\setcounter{pb}{10}
\begin{problem}
    确定下列多项式在有理数域上的分裂域:
    \begin{enumerate}[label=(\roman*)]
        \item $ f(x) = x^4 - 2 $;
        \item $ f(x) = x^3 - 2x - 2 $.
        % \item $ f(x) = x^3 - 3x - 1 $.
    \end{enumerate}
\end{problem}

\begin{solution}
\begin{enumerate}[label=(\roman*)]
    \item 即 $\mathbb{Q}(\sqrt[4]{2},i)$.
    \item 有理根只可能是 $\pm1,\pm2$, 计算可见没有有理根, 从而他是不可约多项式. 判别式 $\Delta=-4(-2)^3-27(-2)^2=-66$ 不是 $\mathbb{Q}$ 中平方元. 
    所以分裂域是 $\mathbb{Q}(\alpha,\sqrt{\Delta})$, 其中 $\alpha$ 是 $f(x)$ 的实数根.
\end{enumerate}
\end{solution}

\subsection{第八章}
\setcounter{pb}{2}
\begin{problem}
    证明域 $ F $ 的每个非零自同态都保持 $ F $ 内素域的元素不动. 设 $ P $ 为含于 $ F $ 内的素域, 于是 $ \operatorname{Aut} F = \operatorname{Gal}(F/P) $.
\end{problem}

\begin{solution}
    设 $\sigma\colon F\to F$ 是非零的自同态, 则 $\sigma$ 是 $F$ 的自同构, 
    所以 $\sigma(1)$ 是单位元即 $\sigma(1)=1$. 于是, $\sigma$ 保持素域内的元素不动, 因为素域由 $1$ 生成. 
    现在 $\operatorname{Aut}F=\operatorname{Gal}(F/P)$ 按照定义直接成立. 
\end{solution}

\setcounter{pb}{4}
\begin{problem}
    确定 $ \operatorname{Gal}(K/\mathbb{Q}) $,其中 $ K = \mathbb{Q}(\sqrt{2}, \sqrt{3}) $.
\end{problem}

\begin{solution}
    记所求为 $G$. 域自同构保持多项式的根集, 所以 $\sqrt{2}\mapsto\sqrt{2}$ 或者 $\sqrt{2}\mapsto-\sqrt{2}$, 且这两种必有一种成立, 同理于 $\sqrt{3}$. 
    令 $\sigma(\sqrt{2})=-\sqrt{2},\sigma(\sqrt{3})=\sqrt{3}$; $\tau(\sqrt{2})=\sqrt{2},\tau(\sqrt{3})=-\sqrt{3}$.
    可知 $G=\langle \sigma,\tau \rangle \cong (\mathbb{Z}/\langle 2 \rangle)^{2}$.
\end{solution}

\setcounter{pb}{6}
\begin{problem}
    设 $ F $ 为多项式环 $ \mathbb{F}_p[t] $ 的商域,即 $ F = \mathbb{F}_p(t) $. 
    令 $ K $ 为多项式 $ f(x) = x^p - t $ 在 $ F $ 上的分裂域. 
    证明 $ \operatorname{Gal}(K/F) = \{1\} $.
\end{problem}

\begin{problem}
    设 $ F = \mathbb{F}_p(t) $ 如习题 6. 
    令 $ K $ 为 $ f(x) = x^{2p} + t x^p + t $ 在 $ F $ 上的分裂域. 
    试决定 $ \operatorname{Gal}(K/F) $,并定出 $ \operatorname{Gal}(K/F) $ 的不动域和 $ F $ 在 $ K $ 内的可分闭包. 
\end{problem}

\setcounter{pb}{14}
\begin{problem}
    设 $ K = \mathbb{Q}(\sqrt{2}, \sqrt{3}) $, $ \theta = (2 - \sqrt{2})(3 + \sqrt{3}) $, $ E = K(\sqrt{\theta}) $.
    证明 $ E/\mathbb{Q} $ 正规,并决定 $ \operatorname{Gal}(E/\mathbb{Q}) $.
\end{problem}

% \begin{solution}
%     这时 $E$ 是 $K[x]$ 中多项式 $f(x)=x^{2}-\theta$ 的分裂域, 所以是正规扩张; $\operatorname{Gal}(E/K)=\{\mathrm{id}_{E},\sigma\}$, 
%     其中 $\sigma(\sqrt{\theta})=-\sqrt{\theta}$.
% \end{solution}

\subsection{补充}
\setcounter{pb}{1}
\begin{problem}
令 $K$ 是有理数 $\mathbb{Q}$ 上全体代数数做成的数域,证明:$K$ 是 $\mathbb{Q}$ 的代数扩张,但不是有限扩张。
\end{problem}

\begin{solution}
    任取 $x\in K$, 由定义, 存在 $f(X)\in\mathbb{Q}[X]$ 使得 $f(x)=0$. 于是是代数扩张. 
    考虑子扩张 $\mathbb{Q}(\{\sqrt[n]{2}\mid n\geq1 \})/\mathbb{Q}$. 
    这不是有限扩张: 注意 $\mathbb{Q}(\{\sqrt[n]{2}\mid n\geq1 \})=\bigcup_{n\geq1}\mathbb{Q}(\sqrt[n]{2})$, 
    如果这是有限扩张, 则一定有 
        \[
            \mathbb{Q}(\{\sqrt[n]{2}\mid n\geq1 \})=\bigcup_{n\in[N]}\mathbb{Q}(\sqrt[n]{2})=\mathbb{Q}(\sqrt[N]{2}),
        \]
    对某个 $N$ 成立. 这不可能, $\sqrt[N+1]{2}$ 不在里面. 综上所述, 这个子扩张无限, 所以问题中的扩张无限. 
\end{solution}

\subsection{更多习题}
\setcounter{pb}{1}
\begin{problem}
Let $F$ be a field of characteristic $p$, $a$ an element of $F$ not of the form $b^p - b$, $b \in F$. Determine the Galois group over $F$ of a splitting field of $x^p - x - a$.
\end{problem}

\begin{solution}
    Let $\beta$ be a root of $f(x)\coloneqq  x^p-x-a$ in the algebraic closure of $F$. We claim that $x^p-x-a$ splits in $F(\beta)$. 
    Notice that $f(x+1)=f(x)$ by the Frobenius endomorphism. Thus, $\beta+1, \beta+2,\dots,\beta+p-1$ are also roots of $f(x)$. 
    The splitting field is $F(\beta)$. There is a automorphism $\sigma$ of $F(\beta)$, determined by $\beta=\beta+1$. 
    We find $\operatorname{ord}\sigma=p$. Thus $\langle \sigma \rangle\cong C_{p}$, the cyclic group of order $p$. 
    From $|\operatorname{Gal}(F(\beta)/F)|=[F(\beta):F]=n$, we have $\langle \sigma \rangle=\operatorname{Gal}(F(\beta)/F)$.
\end{solution}

\setcounter{pb}{21}
\begin{problem}
Let $F$ be a finite field of characteristic $p$ (a prime). Show that $(p - 1) \mid (|F| - 1)$. Hence conclude that if $|F|$ is even then the characteristic is two. (We shall see later that $|F|$ is a power of $p$.)
\end{problem}

\begin{solution}
    The prime field is $\mathbb{F}_{p}\hookrightarrow F$. Then $\mathbb{F}_{p}^{\times}\leq F^{\times}$ as a subgroup. Langrange's Theorem ensures $p-1\mid(|F|-1)$. 
    Thus, $|F|-1=(p-1)k$ for some $k\in\mathbb{Z}$ and thus 
        \[
            |F|=0\pmod 2\implies (p-1)k=1\pmod2.
        \]
    This can happen in the case $p=2$ only.
\end{solution}

% \setcounter{pb}{2}
% \begin{problem}
% Let $I$ be the set of complex numbers of the form $m + n \sqrt{-3}$ where either $m, n \in \mathbb{Z}$ or both $m$ and $n$ are halves of odd integers. Show that $I$ is a subring of $\mathbb{C}$.
% \end{problem}

% \begin{solution}
    
% \end{solution}

\setcounter{pb}{26}
\begin{problem}
    Determine the Galois group $ \operatorname{Gal}(\mathbb{Q}(\sqrt[p]{2}, \zeta_p)/\mathbb{Q}) $.
\end{problem}

\begin{solution}
    Let $G$ denote the Galois group. By the Galois main theorem, we have diagrams:
        \[
            \begin{tikzcd}
                & \mathbb{Q} \arrow[ld, no head] \arrow[rd, no head] &                                         &                             & {\operatorname{Gal}(\mathbb{Q}(\sqrt[p]{2}, \zeta_p)/\mathbb{Q})} \arrow[ld, no head] \arrow[rd, no head] &                         \\
            {\mathbb{Q}(\sqrt[p]2)} \arrow[rd, no head] &                                                    & \mathbb{Q}(\zeta_p) \arrow[ld, no head] & (\mathbb{Z}/\langle {p} \rangle)^{\times}  \arrow[rd, no head] &                                                                                                           & \mathbb{Z}/\langle p \rangle  \arrow[ld, no head] \\
                & {\mathbb{Q}(\sqrt[p]2,\zeta_p)}                    &                                         &                             & \{1\}                                                                                                     &                        
            \end{tikzcd}
        \]
    Let $\mathbb{Z}/\langle {p} \rangle = \langle \tau \rangle $ and $(\mathbb{Z}/\langle p\rangle)^\ast = \langle \sigma \rangle $, where 
        \[
            \tau \colon 
                \begin{cases} 
                \zeta_p \mapsto \zeta_p \\ 
                \sqrt[p]{2}\zeta_p^i \mapsto \sqrt[p]{2}\zeta_p^{i+1} & i\in[p] 
                \end{cases},\quad
            \sigma \colon 
                \begin{cases}
                    \zeta_{p}\mapsto\zeta^{a}_{p}\\
                    \sqrt[p]{2}\mapsto\sqrt[p]{2}
                \end{cases},
        \]
    where $a\in(\mathbb{Z}/\langle p\rangle)^\ast$. 
    Thus we find two subgroups $\mathbb{Z}/\langle p \rangle,(\mathbb{Z}/\langle p\rangle)^\ast  $ of $G$, where $(\mathbb{Z}/\langle p\rangle)^\ast$ is normal 
    (because it is the Galois group of a Galois extension). For the structure of the group: 
    \begin{itemize}
        \item They have trivial intersection: because $\mathbb{Q}(\sqrt[p]{2})\cap\mathbb{Q}(\zeta_{p})=\mathbb{Q}$;
        \item They generate the group $G$ because $\operatorname{Inv}\langle \tau, \sigma \rangle=\mathbb{Q}$. 
        \item They satisfy: $\sigma\tau\sigma^{-1}=\tau^{a}$.
    \end{itemize}
    Above all, we have $G\cong\mathbb{Z}/\langle p \rangle \rtimes (\mathbb{Z}/\langle p\rangle)^\ast$, which is isomorphic to the matrix group: 
        \[
            \Biggl\{ 
            \begin{pmatrix} 
            a & b \\ 
            0 & 1 
            \end{pmatrix} 
            \mid 
            a \in (\mathbb{Z}/\langle p \rangle)^{\ast}, 
            b \in \mathbb{Z}/\langle p \rangle 
            \Biggr\} 
            \subseteq \mathrm{GL}(2, \mathbb{Z}/\langle p \rangle)
        \]
\end{solution}

\setcounter{pb}{27}
\begin{problem}
    Please state the Galois main theorem clearly.
\end{problem}

% % \begin{solution}
%     Let $E/F$ be a finite Galois extension. We have the following results:
%     \begin{enumerate}
%         \item 
%     \end{enumerate}
% % \end{solution}